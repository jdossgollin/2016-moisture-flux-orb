\begin{block}{Probabalistic Prediction}
    To estimate the  probabilities of moisture flux conditional on synoptic features, and of synoptic features conditional on planetary-scale features, we build a two-level Bayesian model.
    This model takes three data blocks:
    \begin{enumerate}
        \item $y_{N \times 1}$, in this case the moisture flux
        \item $X_{N \times J}$, ``local'' factors: here $X_1$ is GMX SST anomaly, $X_2$ is W. Atl. Ridge, and $X_3$ is Eastern U.S. Low
        \item $Z_{N \times k}$, ``global'' factors: here $Z_1$ is PNA and $Z_2$ is NAO
    \end{enumerate}
    Then, the model is formulated as
    \begin{align}
        y_t & \sim \normal \qty(\beta_0 + X'_t \beta, \sigma^2) \qqtext{for} t = 1, \ldots, T \label{eq:yX}\\
        X_{t,j} &\sim \normal \qty(\alpha_{0,j} + Z'_t \alpha_j, \tau_j^2) \qqtext{for} j = 1, \ldots, J \label{eq:XZ}
    \end{align}
    To estimate the parameters $\beta_0, \beta, \sigma, \alpha_0, \alpha, \tau$ we use Hamiltonian Monte Carlo Markov Chain sampling using the Bayesian programming language Stan \cite{Carpenter2016}.
\end{block}
