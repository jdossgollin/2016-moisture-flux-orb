\begin{block}{Conceptual Framework}
    Making credible forecasts of future flood probabilities from sub-seasonal to decadal timescales \cite{Merz2014} requires understanding the cross-timescale dynamics that regulate the joint distribution of the frequency, intensity, and persistence of rainfall.
    Case study \cite[\ie][]{Grams2014} and theory support the view that persistent intense rainfall does not occur randomly, but rather that the large-scale transport \& convergence of moisture is governed by specific circulations mechanisms which are in turn modulated by global-scale circulations and persistent, low-frequency boundary conditions.

    For example, the April 2011 flooding in the Ohio-Mississippi river system led to sever damage.
    As \cref{fig:apr2011} shows, the monthly rainfall was driven by a strong and persistent jet which steered multiple cyclones along highly similar trajectories.
    \begin{figure}
        \begin{tikzpicture}[node distance = 1.15cm, auto, font=\sffamily]
	% Place nodes
	\node [action] (boundary) {Boundary Forcings};
	\node [object, below= of boundary] (circulation) {Circulation Anomaly};
	\node [object, below=of circulation] (steering) {Moisture Steering};
	\node [action, below= of steering] (flood) {High Flood Potential};
	% Draw edges
	\path [arrow] (boundary) -- (circulation);
    \path [arrow] (circulation) -- (steering);
    \path [arrow] (steering) -- (flood);
\end{tikzpicture}
~\hfill
        \includegraphics[width=0.55\columnwidth]{flooding_2011_flooding}
        \caption{(L) Mechanistic casual hierarchy for extreme, regional floods. (R) Tracked cyclones (lines) and monthly-mean \SI{250}{\hecto\pascal} winds for April 2011. Approximate region affected is circled in blue.}
        \label{fig:apr2011}
    \end{figure}
\end{block}
