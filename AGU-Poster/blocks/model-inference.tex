\begin{block}{Model Inferences}
    \begin{figure}
        \centering
        \includegraphics[width=0.95\columnwidth]{bayesian_posterior}
        \caption{\num{4000} draws from posterior distribution specified by \cref{eq:yX,eq:XZ}.}
        \label{fig:posterior}
    \end{figure}
    Interpreting the results of \cref{fig:posterior} requires noting the ordering of the $X$ and $Z$ variables; for example, \texttt{alpha[1,3]} gives the coefficient of $Z_{1}$ (PNA) on $X_3$ (GMX SSTs).
    In the first (local-moisture) step, we note that the effect of the dynamical $X$ variables (the low and high height anomalies; $\beta_1, \beta_2$) is far greater than the effect of the thermodynamic variable (the GMX SSTs; $\beta_3$).
    This comparison is valid because all $X$ and $Z$ have been standardized.
    Other variables should be explored before drawing more general conclusions.
    It is also of note that the magnitude of the effect of increasing the W. Atlantic High ($\beta_1$) is somewhat larger than increasing the magnigude of the Eastern U.S. Low $(\beta_2)$.
    In the second (global-local) step, we see that a positive PNA provides feedbacks that both support and inhibit moisture flux to the Ohio River Basin, suppressing the high and GMX SSTs ($\alpha_{1,1}, \alpha_{1,3}$) but enhancing the low ($\alpha_{1,2}$) in the positive phase.
    The NAO also provides both moisture-inhibiting and moisture-enhancing effects ($\alpha_{2,1:3}$).
\end{block}
