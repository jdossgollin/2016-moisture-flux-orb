\begin{block}{Large-Scale Modulation of Synoptic Mechanisms}
    The location of extratropical cyclones modulates moisture flux into the Ohio River Basin.
    To see this, for each time step we take the cyclone closest to the region and assign it to the observed moisture flux for that time step.
    \Cref{fig:track-given-flux} (L) shows the results of a local regression \cite{Loader1999} fit to observed data.
    The statistical model allows us to assign a score from zero to one for the (rescaled) expected moisture flux into the region, given an arbitrary cyclone location.

    The right hand figure further demonstrates that cyclone tracks associated with extremely high flux follow highly similar trajectories.
    \emph{The similarity of these trajectories suggests planetary-scale, steering mechanisms.}
    These tracks also resemble those shown in \cref{fig:apr2011} that led to flooding in April 2011.
    \begin{figure}
        \includegraphics[width=0.45\columnwidth]{locfit_weight_predicted_observed}~
        \includegraphics[width=0.5\columnwidth]{moisture_cyclone_tracks_given_flux}
        \caption{(L) Expected moisture transport given cyclone location. Units are normalized from zero to one. Colors show observation and contours show the statistical model.
                (R) Tracks of extratropical cyclones associated with (L) 70 random DJF days, and (R) the 70 days with highest DJF moisture transport. The ``Moisture'' box of \cref{fig:study-area} is shown for reference.}
        \label{fig:track-given-flux}
    \end{figure}
\end{block}
