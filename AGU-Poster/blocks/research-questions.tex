\begin{block}{Research Questions \& Data Used}
    \begin{enumerate}
        \item What synoptic weather patterns lead to intense moisture transport into the Ohio River Basin?
        \item How do planetary-scale circulations modulate these synoptic weather patterns?
        \item What are the conditional probabilities of moisture transport into the region, given synoptic and planetary circulation indices?
        \item How can predictions of low-frequency modes of planetary circulation inform risk of extreme, regional flooding?
    \end{enumerate}
    We use the Ohio River Basin (\cref{fig:study-area}) to anchor a case study, focusing on the DJF season during which the well-known 1937 flood occurred.
    We use 6-hourly reanalysis data from ERA-Interim \cite{Dee2011} (1979-2015) and cyclone tracks generated by Donna Lee \cite{Booth2015} using the tracking algorithm of \cite{Hodges1994}.
\end{block}
