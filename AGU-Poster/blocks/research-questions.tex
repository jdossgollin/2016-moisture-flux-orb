\begin{block}{Research Questions \& Data Used}
    Using  the Ohio River Basin (\cref{fig:study-area}) as a case study, we focus on DJF in order to reduce the effects of tropical cyclones and mesocale convection.
    There is also relvance for flood impacts: the well-known 1937 flood occurred during January and February.
     We use 6-hourly reanalysis data from ERA-Interim \cite{Dee2011} (1979-2015) and cyclone tracks from \cite{Booth2015} using the tracking algorithm of \cite{Hodges1994} to address the following questions:
    \begin{enumerate}
        \item What synoptic weather patterns lead to intense moisture transport into the Ohio River Basin?
        \item How do planetary-scale circulations modulate these synoptic weather patterns?
        \item What are the conditional probabilities of moisture transport into the region, given synoptic and planetary circulation indices?
    \end{enumerate}
    %These questions correspond to the blue boxes highlighted in \cref{fig:apr2011}.
\end{block}
