\documentclass[serif,mathserif,final]{beamer}
\mode<presentation>{\usetheme{Lankton}}


\setbeamertemplate{caption}[numbered] % number figs
\usepackage[
  orientation=landscape,
  size=custom,
  width=152.400,height=76.200,%size=a0, % paper size
  scale=1.1 % font scale factor
  ]{beamerposter}


% Graphics Path
\usepackage{graphicx}
\graphicspath{{"../figs/"}}

% OTHER PACKAGE CALLS
\usepackage[utf8]{inputenc}
\usepackage{amsmath,amssymb,physics} % math symbols
\usepackage{siunitx}
\sisetup{round-mode = figures, round-precision = 3}
\usepackage{tikz}
\usetikzlibrary{
	shapes,
    arrows,
    shapes.geometric,
    positioning,
}

% Define block styles
\tikzstyle{object} = [
	rectangle,
    draw,
    fill=blue!40,
    text width=9cm,
    text badly centered,
    rounded corners]

\tikzstyle{action} = [
	rectangle,
	draw,
    text width=9cm,
    text centered,
    rounded corners]

\tikzstyle{arrow} = [draw, -latex']
\tikzstyle{noarrow} = [draw, dashed]

% set consistent font size
\tikzset{every picture/.style={font issue=\footnotesize},
         font issue/.style={execute at begin picture={#1\selectfont}}
         }


% eg and ie
\usepackage{xspace}
\newcommand*{\eg}{e.g.\@\xspace}
\newcommand*{\ie}{i.e.\@\xspace}

% this package loads last except for biblatex
\usepackage{cleveref}

% Use BibLaTeX
\usepackage[
	backend=biber,
	doi=false,
	url=false,
	isbn=false,
	maxnames=3,
	minnames = 1,
	style=numeric-comp,
	sorting=none,
	natbib=true,
	firstinits=true,
	isbn=false
  ]{biblatex}
\renewbibmacro{in:}{}
\addbibresource{../library.bib}
\AtEveryBibitem{
  \clearfield{month}
  \clearlist {language}
  \clearfield{pages}
  \clearfield{pages}
  \clearfield{volume}
  \clearfield{number}
  \clearfield{title}
}
\setbeamertemplate{bibliography item}[text] % don't print the silly symbols

%-- Header and footer information ----------------------------------
\newcommand{\footleft}{\url{water.columbia.edu}}
\newcommand{\footright}{\url{james.doss-gollin@columbia.edu}}
\title{Global-Local Interactions Modulate Tropical Moisture Export to the Ohio River Basin}
\author{James Doss-Gollin\inst{1,2} \quad David Farnham\inst{1,2} \quad Upmanu Lall\inst{1,2}}
\institute
{\inst{1} Columbia Water Center \quad \inst{2} Department of Earth and Environmental Engineering, Columbia University}

%-------------------------------------------------------------------


%-- Main Document --------------------------------------------------
\begin{document}

% Statistical Macros
\def\ci{\perp\!\!\!\perp}
\def\ex{\mathbb{E}}
\def\prob{\mathbb{P}}
\def\ind{\mathbb{I}}
\def\grad{\triangledown}
\def\bigo{\mathcal{O}}
\def\normal{\mathcal{N}}
\def\bern{\text{Bernoulli}}
\def\logit{\text{logit}}
\def\binom{\text{Bin}}
\def\poiss{\text{Poiss}}
\def\cauchy{\text{Cauchy}}
\def\sigmoid{\vb{\sigma}}
\def\given{\big|}
\def\stan{\texttt{Stan~}}

%A little bit on why this region is interesting & our broad hypotheses (floods come from a hierarchy of causal mechanisms)
%Establish that synoptic features modulate moisture flux -- I think this is already done with the maps that I have of flux given cyclone positions
%Establish that "global" features (i.e. PNA, AMO) modulate these synoptic features -- again I think plots of cyclone tracks given different PNA phases is good
%Explain that the purpose of the Bayesian model is to learn how the large-scale features modulate the local features
%Present the model very briefly
%Finish by taking a subset of the data, predicting using the observed values of PNA + AMO, and comparing to just regressing on PNA & AMO with no intermediate step.
\begin{frame}{}
  \begin{columns}[t]

    %-- Column 1 ---------------------------------------------------
    \begin{column}{0.18\linewidth}

        \begin{block}{Conceptual Framework}
    Making credible forecasts of future flood probabilities from sub-seasonal to decadal timescales \cite{Merz2014} requires understanding the cross-timescale dynamics that regulate the joint distribution of the frequency, intensity, and persistence of rainfall.
    Case study \cite[\ie][]{Grams2014} and theory support the view that persistent intense rainfall does not occur randomly, but rather that the large-scale transport \& convergence of moisture is governed by specific circulations mechanisms which are in turn modulated by global-scale circulations and persistent, low-frequency boundary conditions.

    For example, the April 2011 flooding in the Ohio-Mississippi river system led to sever damage.
    As \cref{fig:apr2011} shows, the monthly rainfall was driven by a strong and persistent jet which steered multiple cyclones along highly similar trajectories.
    \begin{figure}
        \begin{tikzpicture}[node distance = 1.15cm, auto, font=\sffamily]
	% Place nodes
	\node [action] (boundary) {Boundary Forcings};
	\node [object, below= of boundary] (circulation) {Circulation Anomaly};
	\node [object, below=of circulation] (steering) {Moisture Steering};
	\node [action, below= of steering] (flood) {High Flood Potential};
	% Draw edges
	\path [arrow] (boundary) -- (circulation);
    \path [arrow] (circulation) -- (steering);
    \path [arrow] (steering) -- (flood);
\end{tikzpicture}
~\hfill
        \includegraphics[width=0.55\columnwidth]{flooding_2011_flooding}
        \caption{(L) Mechanistic casual hierarchy for extreme, regional floods. (R) Tracked cyclones (lines) and monthly-mean \SI{250}{\hecto\pascal} winds for April 2011. Approximate region affected is circled in blue.}
        \label{fig:apr2011}
    \end{figure}
\end{block}

        \begin{block}{Research Questions}
    Moisture flux can be more predictable than extreme rainfall \cite{Lavers2016}.
    Here, we address the following questions, using the Ohio River Basin (\cref{fig:study-area}) as a case study and focusing on the DJF season because other mechanisms of moisture transport are suppressed:
    \begin{enumerate}
        \item What synoptic circulations lead to intense tropical moisture export into the Ohio River Basin?
        \item What planetary-scale circulations modulate these synoptic circulations?
        \item What are the conditional probabilities of moisture given synoptic and planetary circulations?
    \end{enumerate}
    \begin{figure}
        \centering
        \includegraphics[width=0.9\columnwidth]{map_inset}
        \caption{Study Area. Shaded area shows the Ohio River Basin. Boxes show area over which moisture flux is calculated, Western Atlantic Ridge, and Gulf of Mexico SST region.}
        \label{fig:study-area}
    \end{figure}
\end{block}

        \begin{block}{Acknowledgements}
    Thanks to Donna Lee for sharing cyclone tracks.
    Thanks to Yochanan Kushnir, Jimmy Booth, Pierre Gentine, Casey Brown, Linda Mearns, Katherine Schlef, Melissa Bukovsky, Rachel McCrary, and Seth McGinnis for conversation and guidance.
    This project is funded by the Department of Defense grant \# SERDP 15 RC02-060.
\end{block}


    \end{column}%1

    %-- Column 2 ---------------------------------------------------
    \begin{column}{0.18\linewidth}
        \begin{block}{Study Area \& Variable Definitions}
    \begin{figure}
        \centering
        \caption{Composite anomalies of \SI{500}{\hecto\pascal} geopotential height (color) and precipitable water (contours) from 4 days preceding regional intense precipitation days in the Ohio River Basin to one day following.}
        \includegraphics[width=\columnwidth]{../FigExternal/djf_composites}
        \label{fig:djf-composites}
    \end{figure}
    Study of regional intense precipitation events in the Ohio River Basin (\cref{fig:djf-composites}) reveals the dominance of a ridge (somtimes stationary, sometimes transient) over the Western Atlantic and a transient cyclone propagating eastward.
    To study this, we define the following variables, separated into planetary-scale and synoptic-scale:
    \begin{description}
        \item[Moisture] the daily net moisture flux into Ohio River Basin region (\cref{fig:study-area}, green box)
        \item[Planetary] 30-day moving average of the PNA and NAO indices from the CPC
        \item[Synoptic] Mean SST anomaly over the Gulf of Mexico (\cref{fig:study-area}, blue); and the \SI{850}{\hecto\pascal} the geopotential height anomalies over the West Atlantic (\cref{fig:study-area}, purple) and the Eastern United States (red).
    \end{description}
    \begin{figure}
        \centering
        \includegraphics[width=0.8\columnwidth]{map_inset}
        \caption{Study Area. Shaded area shows the Ohio River Basin.}
        \label{fig:study-area}
    \end{figure}
\end{block}


    \end{column}%2

    %-- Column 3 ---------------------------------------------------
    \begin{column}{0.18\linewidth}
        \begin{block}{Synoptic Patterns}
    Inspection of cyclone track locations and the associated moisture flux into the region (not shown) shows that moisture flux into the region is maximized in the presence of a cyclone to the North-West of the region.
    \Cref{fig:track-given-flux} shows this in terms of cyclone tracks: tracks associated with extremely high flux (R) follow a clear preferential trajectory from the South-West to North-East, a much less frequently observed pattern among randomly sampled tracks (L).
    \begin{figure}
        \includegraphics[width=0.9\columnwidth]{moisture_cyclone_tracks_given_flux}
        \caption{Tracks of extratropical cyclones associated with (L) 70 random DJF days, and (R) the 70 days with highest DJF moisture transport. Color indicates cyclone intensity.}
        \label{fig:track-given-flux}
    \end{figure}
\end{block}

        \begin{block}{Global-Scale Modulation}
    A primary mechanism by which global-scale circulations such as the PNA modulate moisture flux into the mid-latitudes is by shifting the storm tracks.
    \Cref{fig:track-given-pna} shows the cyclone tracks over the study area for the negative and positive PNA terciles.
    Relative to the negative phase, the positive PNA shifts the SW$\rightarrow$NE tracks to the east and favors a NW$\rightarrow$SE trajectory over the Ohio River Basin.
    \begin{figure}
        \includegraphics[width=\columnwidth]{pna_cyclone_map_plot}
        \caption{Cyclone tracks given PNA in (L) negative; (R) positive terciles (neutral omitted)}
        \label{fig:track-given-pna}
    \end{figure}
\end{block}


    \end{column}%3

    %-- Column 4 ---------------------------------------------------
    \begin{column}{0.18\linewidth}

      \begin{block}{Probability Model}
    To estimate the  probabilities of moisture flux conditional on synoptic features, and of synoptic features conditional on planetary-scale features, we build a two-level Bayesian model.
    This model takes three fundamental data blocks:
    \begin{enumerate}
        \item $y_{N \times 1}$, in this case the moisture flux
        \item $X_{N \times J}$, the ``local'' factors, in this case the GMX SSTs, W. Atl. Ridge, and Eastern U.S. Low
        \item $Z_{N \times k}$, the ``global'' factors, in this case the PNA and NAO
    \end{enumerate}
    Then, the model is formulated as
    \begin{align}
        y_t & \sim \normal \qty(\beta_0 + X'_t \beta, \sigma^2) \qqtext{for} t = 1, \ldots, T \label{eq:yX}\\
        X_{t,j} &\sim \normal \qty(\alpha_{0,j} + Z'_t \alpha_j, \tau_j^2) \qqtext{for} j = 1, \ldots, J \label{eq:XZ}
    \end{align}
    To estimate the parameters $\beta_0, \beta, \sigma, \alpha_0, \alpha, \tau$ we use Hamiltonian Monte Carlo Markov Chain sampling using the Bayesian programming language Stan \cite{Carpenter2016}, applying noninformative priors \cite{Gelman2014}.
\end{block}


    \end{column}%3


    %-- Column 5 ---------------------------------------------------
    \begin{column}{0.18\linewidth}
        \begin{block}{Summary of Findings}
    \begin{enumerate}
        \item A finite set of specific synoptic circulation patterns accounts for most DJF moisture transport into the Ohio River Basin
        \item Persistent steering mechanisms such as the PNA and NAO alter the dominant directions of cyclone propagation over the region, leading to changes in moisture transport
        \item The PNA and NAO provide both positive and negative feedbacks on moisture transport to the Ohio River Basin. Modeling these intermediate feedbacks explicity \textbf{improves model propagation of opposing feedbacks and variance} as compared to modeling moisture transport directly on the global-scale indices.
    \end{enumerate}
\end{block}

      %-- Block 4-2
      \begin{block}{References}
          \renewcommand*{\bibfont}{\footnotesize}
          \printbibliography[heading=none]
      \end{block}


    \end{column}%1

  \end{columns}
\end{frame}
\end{document}
