\documentclass[serif,mathserif,final]{beamer}
\mode<presentation>{\usetheme{Lankton}}


\setbeamertemplate{caption}[numbered] % number figs
\usepackage[
  orientation=landscape,
  size=custom,
  width=101.600,height=76.200,%size=a0, % paper size
  scale=1.0 % font scale factor
  ]{beamerposter}


% Graphics Path
\usepackage{graphicx}
\graphicspath{{"../figs/"}}

% OTHER PACKAGE CALLS
\usepackage[utf8]{inputenc}
\usepackage{amsmath,amssymb,xspace}
\usepackage{siunitx}
\sisetup{round-mode = figures, round-precision = 3}
\usepackage{cleveref}
\usepackage{tikz}
\usetikzlibrary{
	shapes,
    arrows,
    shapes.geometric,
    positioning,
}

% Define block styles
\tikzstyle{object} = [
	rectangle,
    draw,
    fill=blue!20,
    text width=2.5cm,
    text badly centered,
    rounded corners]

\tikzstyle{action} = [
	rectangle,
	draw,
    text width=4.5cm,
    text centered,
    rounded corners]

\tikzstyle{arrow} = [draw, -latex']
\tikzstyle{noarrow} = [draw, dashed]

% set consistent font size
%\tikzset{every picture/.style={font issue=\footnotesize},
%         font issue/.style={execute at begin picture={#1\selectfont}}
%         }


\usepackage{xspace}
\newcommand*{\eg}{e.g.\@\xspace}
\newcommand*{\ie}{i.e.\@\xspace}

% Use BibLaTeX
\usepackage[
	backend=biber,
	doi=false,
	url=false,
	isbn=false,
	maxnames=3,
	minnames = 1,
	style=numeric-comp,
	sorting=none,
	natbib=true,
	firstinits=true,
	isbn=false
  ]{biblatex}
\renewbibmacro{in:}{}
\addbibresource{../library.bib}
\AtEveryBibitem{
  \clearfield{month}
  \clearlist {language}
  \clearfield{pages}
  \clearfield{pages}
  \clearfield{volume}
  \clearfield{number}
  \clearfield{title}
}
\setbeamertemplate{bibliography item}[text] % don't print the silly symbols

%-- Header and footer information ----------------------------------
\newcommand{\footleft}{\url{water.columbia.edu}}
\newcommand{\footright}{\url{james.doss-gollin@columbia.edu}}
\title{}
\author{James Doss-Gollin\inst{1,2} \quad David Farnham\inst{1,2} \quad Upmanu Lall\inst{1,2}}
\institute
{\inst{1} Columbia Water Center \quad \inst{2} Department of Earth and Environmental Engineering, Columbia University}

%-------------------------------------------------------------------


%-- Main Document --------------------------------------------------
\begin{document}
%A little bit on why this region is interesting & our broad hypotheses (floods come from a hierarchy of causal mechanisms)
%Establish that synoptic features modulate moisture flux -- I think this is already done with the maps that I have of flux given cyclone positions
%Establish that "global" features (i.e. PNA, AMO) modulate these synoptic features -- again I think plots of cyclone tracks given different PNA phases is good
%Explain that the purpose of the Bayesian model is to learn how the large-scale features modulate the local features
%Present the model very briefly
%Finish by taking a subset of the data, predicting using the observed values of PNA + AMO, and comparing to just regressing on PNA & AMO with no intermediate step.
\begin{frame}{}
  \begin{columns}[t]

    %-- Column 1 ---------------------------------------------------
    \begin{column}{0.22\linewidth}

        % Example Block
        \begin{block}{Conceptual Framework}
    Making credible forecasts of future flood probabilities from sub-seasonal to decadal timescales \cite{Merz2014} requires understanding the cross-timescale dynamics that regulate the joint distribution of the frequency, intensity, and persistence of rainfall.
    Case study \cite[\ie][]{Grams2014} and theory support the view that persistent intense rainfall does not occur randomly, but rather that the large-scale transport \& convergence of moisture is governed by specific circulations mechanisms which are in turn modulated by global-scale circulations and persistent, low-frequency boundary conditions.

    For example, the April 2011 flooding in the Ohio-Mississippi river system led to sever damage.
    As \cref{fig:apr2011} shows, the monthly rainfall was driven by a strong and persistent jet which steered multiple cyclones along highly similar trajectories.
    \begin{figure}
        \begin{tikzpicture}[node distance = 1.cm, auto, font=\sffamily]
	% Place nodes
	\node [action] (boundary) {Boundary Forcings\\ (\ie ENSO)};
	\node [object, below= of boundary] (circulation) {Circulation Anomaly\\ (\ie PNA)};
	\node [object, below=of circulation] (steering) {Moisture Steering\\ (\ie Cyclone)};
	\node [action, below= of steering] (flood) {High Flood Potential};
	% Draw edges
	\path [arrow] (boundary) -- (circulation);
    \path [arrow] (circulation) -- (steering);
    \path [arrow] (steering) -- (flood);
\end{tikzpicture}
~\hfill
        \includegraphics[width=0.6\columnwidth]{flooding_2011_flooding}
        \caption{(L) Mechanistic casual hierarchy for extreme, regional floods. (R) Tracked cyclones (lines) and monthly-mean \SI{250}{\hecto\pascal} winds for April 2011.}
        \label{fig:apr2011}
    \end{figure}
\end{block}

        \begin{block}{Data Used}
    Reanalysis data from ERA-Interim \cite{Dee2011}.
    Cyclone tracks courtesy of Donna Lee; see \cite{Booth2015}.
\end{block}


    \end{column}%1

    %-- Column 2 ---------------------------------------------------
    \begin{column}{0.22\linewidth}

        % Example Block
        \begin{block}{Rainfall Sequences}
    \begin{figure}[h]
        \caption{A caption}
        \includegraphics[width=.9\columnwidth]{pna_cyclone_map_plot}
        \label{fig:example}
    \end{figure}
    A text
    \begin{enumerate}
        \item Enhanced SALLJ activity
        \item Enhanced low-level moisture transport from Atlantic
    \end{enumerate}
\end{block}


    \end{column}%2

    %-- Column 3 ---------------------------------------------------
    \begin{column}{0.22\linewidth}

      % Example Block
      \begin{block}{Rainfall Sequences}
    \begin{figure}[h]
        \caption{A caption}
        \includegraphics[width=.9\columnwidth]{pna_cyclone_map_plot}
        \label{fig:example}
    \end{figure}
    A text
    \begin{enumerate}
        \item Enhanced SALLJ activity
        \item Enhanced low-level moisture transport from Atlantic
    \end{enumerate}
\end{block}


    \end{column}%3


    %-- Column 4 ---------------------------------------------------
    \begin{column}{0.22\linewidth}

      %-- Block 4-2
      \begin{block}{References}
          \renewcommand*{\bibfont}{\footnotesize}
          \printbibliography[heading=none]
      \end{block}

      \begin{block}{Acknowledgements}
          Thanks to Yochanan Kushnir, Jimmy Booth, Donna Lee
      \end{block}

    \end{column}%1

  \end{columns}
\end{frame}
\end{document}
